\documentclass{fhw}

\usepackage{blindtext}

\title{Beautiful Homework}
\author{Author}
\ID{b11902017}
\CourseID{CSIE5302}
\CourseName{\LaTeX\;Template}

% Toggle this for better BW printing
\NoPrint
% \YesPrint

\begin{document}

\maketitle

\section{Basic Advanced Usage}

\subsection{Enumerate}

Enumerate can be used like this:
\begin{Code}{tex}
\begin{enumerate}[(a)]
  \item one
  \item deux
    \setcounter{enumi}{4}
  \item fünf
    \begin{enumerate}[i.]
      \item and you can
      \item nest them.
    \end{enumerate}
  \item[item6] Custom item.
\end{enumerate}
\end{Code}

The code will become: 

\begin{enumerate}[(a)]
	\item 一
	\item deux
	\setcounter{enumi}{4}
	\item fünf
		\begin{enumerate}[i.]
			\item and you can
			\item nest them.
		\end{enumerate}
	\item[item6] Custom item label.
\end{enumerate}

\subsubsection{Notes}

I often uses \mintinline{tex}|\setlength{\itemsep}{0pt}\setlength{\parskip}{0pt}| to make itemizes and enumerates tighter.
Here's what the above list looks like with it:

\begin{enumerate}[(a)]\setlength{\itemsep}{0pt}\setlength{\parskip}{0pt}
	\item 一
	\item deux
	\setcounter{enumi}{4}
	\item fünf
		\begin{enumerate}[i.]
			\item and you can
			\item nest them.
		\end{enumerate}
	\item[item6] Custom item label.
\end{enumerate}

\subsection{Math symbols}

There are some new symbols defined in this template:

\begin{Code}{tex}
\[
\N \Z \Q \R \C \contra
\abs{a}, \ceil{b}, \floor{c}, \inner{x}, \norm{y}, \set{1, 2, \frac{3}{7}}
a * b = c, x \ast y = z
\]
\end{Code}

\[
\N \Z \Q \R \C \contra
\abs{a}, \ceil{b}, \floor{c}, \inner{x}, \norm{y}, \set{1, 2, \frac{3}{7}}
a * b = c, x \ast y = z
\]

You can also use \verb|*| for product, no more \verb|\cdot|!

If you want to have equations with multiple lines and be aligned, you can use \verb|align*|:

\begin{align*}
  & \int_{0}^{3} \abs{v\left(t\right)} dt \\
  & = \int_{0}^{1} -v\left(t\right) dt + \int_{1}^{3} v\left(t\right) dt \\
  & = \eval{\left(\frac{1}{3}t^3 + \frac{3}{2} t^2 - 4t\right)} _ {1} ^ {0}
  + \eval{\left(\frac{1}{3}t^3 + \frac{3}{2} t^2 - 4t\right)} _ {1} ^ {3} \\
  & = \left[0 - \left(\frac{1}{3} + \frac{3}{2} - 4\right)\right] + \left[\left(9 + \frac{27}{2} - 12\right) - \left(\frac{1}{3} + \frac{3}{2} - 4\right)\right] \\
  & = \frac{89}{6}
\end{align*}

Remember to use \verb|\left( x \right)| for parantheses autoscaling, and \verb|\limits| to put things on top of / under $\sum$ and more! More examples:

\[
  \abs{1 + \frac xy}^2
\]
\[
	\abs{\frac{1}{1 - \lambda h}} \le 1
	\qquad\text{and}\qquad
	\bigcup_{i=1}^n \; \set{z \in \C \mid \abs*{z - a_{ii}} \le {\sum\limits_{j \ne i}} \abs*{a_{ij}}}.
\]

\subsection{Text ornaments}

\begin{Code}{tex}
\textbf{Bold text}, \textit{Italic \footnote{Actually it's oblique.} text} with footnote or \textsl{Slanted text}, \texttt{Teletype text}, \emph{Emphasis with underline}, \underline{Underlined}, \sout{Strike through} work like this.
\end{Code}

\textbf{Bold text}, \textit{Italic \footnote{Actually it's oblique.} text} with footnote or \textsl{Slanted text}, \texttt{Teletype text}, \emph{Emphasis with underline}, \underline{Underlined}, \sout{Strike through} work like this.

\subsection{URLs and hyperrefs}

\begin{Code}{tex}
URLs like this: \url{https://tex.stackexchange.com/questions/61015/how-to-use-different-colors-for-different-href-commands}\\
\href{https://en.wikibooks.org/wiki/LaTeX/Counters}{And hyperref like this.}
\end{Code}
URLs like this: \url{https://tex.stackexchange.com/questions/61015/how-to-use-different-colors-for-different-href-commands}\\
\href{https://en.wikibooks.org/wiki/LaTeX/Counters}{And hyperref like this.}

\section{Blocks}

\subsection{Theorems, Lemmas, ...}

\begin{Code}{tex}
\begin{theorem}
  Theorem 1. \blindtext
\end{theorem}

\begin{lemma}
  Lemma 1. They have seperate numbers.
\end{lemma}

\begin{claim}
  Claim for DSA proof!
\end{claim}

\begin{observation}
  My observation. You can make \verb|\label{}|s...
  \label{myob}
\end{observation}

\begin{lemma}[My lemma]
  ...give them names, and reference them like this:\\
  According to \cref{myob}...
\end{lemma}
\end{Code}

\begin{theorem}
	Theorem 1. \blindtext
\end{theorem}

\begin{lemma}
	Lemma 1. They have seperate numbers.
\end{lemma}

\begin{claim}
  Claim for DSA proof!
\end{claim}

\begin{observation}
  My observation. You can make \verb|\label{}|s...
	\label{myob}
\end{observation}

\begin{lemma}[My lemma]
	...give them names, and reference them like this:\\
	According to \cref{myob}...
\end{lemma}

\subsection{Problems!}

This might be the most useful command of all!
\begin{Code}{tex}
\begin{problem}
  This is a problem.
\end{problem}
\problem You can also use without \verb|\begin| and \verb|\end|.
\problem* If you don't want \verb|\problem| to start on a new page, you can add an asterisk after them. It's unlike other commands, where adding an asterisk disables numbering.
\problem*[C8763 --- Starburst] You can also number them yourself!
\problem* Problem numbering is independent, too.
\end{Code}

\begin{problem}
  This is a problem.
\end{problem}
\problem You can also use without \verb|\begin| and \verb|\end|.
\problem* If you don't want \verb|\problem| to start on a new page, you can add an asterisk after them. It's unlike other commands, where adding an asterisk disables numbering.
\problem*[C8763 --- Starburst] You can also number them yourself!
\problem* Problem numbering is independent, too.

Why do problems start on new page? Because it's more convenient when dealing with GradeScope!

\subsection{Code Blocks}

You can input code files like this:

\begin{Code}{tex}
% \CodeFile[minted options]{lang}{filename}{caption}{label}
\CodeFile[firstline=2]{cpp}{a.cpp}{Example code}{ex}
And refer to them as \cref{code:ex}!
\end{Code}

\CodeFile[firstline=2]{cpp}{a.cpp}{Example code}{ex}
And refer to them as \cref{code:ex}!

Or write code inside the tex file like this:
\begin{Code}{tex}
\begin{Codee}{cpp}
#include <iostream>
int main() {}
\end{Codee} % extra e because it won't compile if there's two \end{Code}
\end{Code}

\begin{Code}{cpp}
#include <iostream>
int main() {}
\end{Code}

% This is an inline code test: \InlineCode{#include <iostream>}
\mintinline{cpp}{for (int i = 0; i < 'a'; ++i) cout << zisk; }

\begin{algorithm}
\begin{algorithmic}[1]
  \Procedure{Get-Min-Index}{A}
  \State m = 1
  \For {i = 2 to $A$.length}
    \State // update if i-th element smaller
    \If { $A[m] > A[i]$ }
      \State $m = i$
    \EndIf
  \EndFor
  \State \Return $m$
  \EndProcedure
\end{algorithmic}
\end{algorithm}

\subsubsection{Notes}

Though I made the Code environment, I almost always uses this instead:
\begin{Code}{tex}
\begin{spacing}{1}
  % some file from my CS course
  \inputminted{py}{lab1/stackoverflow/solve.py}
\end{spacing}
\end{Code}

\subsection{Images}

Though you can manually add images by \verb|\includegraphics|, I've made a command here:
\begin{Code}{tex}
% \Image[size in \textwidth's default 0.8]{filename}{caption}{label}
\Image[0.7]{cactus.jpeg}{Cactus!}{cactus}
And refer to them like \cref{img:cactus}.
\end{Code}

% \Image[size in \textwidth's default 0.8]{filename}{caption}{label}
\Image[0.7]{cactus.jpeg}{Cactus!}{cactus}
And refer to them like \cref{img:cactus}.

\section{TikZ}

I don't know how to teach TikZ in a short-enough-to-fit-in-this-document length, so here's one example:

\CodeFile{tex}{omega.tex}{TikZ example}{tikzex}

It looks like this:

\Image[0.3]{omega.pdf}{TikZ example result}{tikzexr}

I always compile them as different files and input them with figures. However, they can also be done inside the same tex file.

Some good example TikZ graph from IOICamp2024 Flow course for you to steal:

\tikzset{vertex/.style={circle, draw, thick, minimum size=.8cm}}
\tikzset{source/.style={vertex, fill=black!20}}
\tikzset{fed/.style={draw, -latex, thick}}
\tikzset{fat/.style={draw, -latex, line width=0.8mm}}
\tikzset{%
  apply style/.code={%
    \tikzset{#1}%
  }
}
\begin{figure}[H]
  \centering
  \begin{tikzpicture}[x=1.2cm,y=-1.2cm]
    \node[source](s) at (0, 0) {$s$};
    \node[source](t) at (0, 3) {$t$};
    \foreach \x/\y/\name/\lab in {
      -1.4/0.7/$v_2$/v2,
      1.8/0.7/$v_3$/v3,
      0.3/1.4/$v_4$/v4,
      1.9/2.3/$v_5$/v5,
      -1.4/2.5/$v_6$/v6%
    } {
      \node[vertex] (\lab) at (\x, \y) {\name};
    }
    \begin{scope}[every edge/.append style={pos=.5}]
      \foreach \u/\v/\f/\ff/\style in {%
        s/v2/4/2/{sloped, above},
        s/v4/6/4/{left},
        s/v3/1/1/{sloped, above},
        s/v5/3/3/{left, pos=.8},
        v2/v4/2/0/{sloped, below},
        v2/v6/2/2/{left},
        v6/t/3/2/{sloped, above,pos=.4},
        v4/t/5/4/{left, pos=.4},
        v5/t/4/4/{sloped, above,pos=.4},
        v3/v5/4/1/{right}%
      }{
        \draw[fed] (\u) edge[apply style/.expand once=\style] 
          node{{\bf\Large\ff}{/\f}} (\v);
      }
    \end{scope}
  \end{tikzpicture}
  \begin{tikzpicture}[x=1.2cm,y=-1.2cm]
    \node[source](s) at (0, 0) {$s$};
    \node[source](t) at (0, 3) {$t$};
    \foreach \x/\y/\name/\lab in {
      -1.4/0.7/$v_2$/v2,
      1.8/0.7/$v_3$/v3,
      0.3/1.4/$v_4$/v4,
      1.9/2.3/$v_5$/v5,
      -1.4/2.5/$v_6$/v6%
    } {
      \node[vertex] (\lab) at (\x, \y) {\name};
    }
    \begin{scope}[every edge/.append style={pos=.5}]
      \foreach \u/\v/\f/\style in {%
        s/v2/2/{sloped,above,bend right=60,line width=0.1cm},
        v2/s/2/{sloped,above},
        s/v4/2/{left,bend right},
        v4/s/4/{right,pos=.3},
        v3/s/1/{sloped, above},
        v5/s/3/{left, pos=.2},
        v2/v4/2/{sloped,below,bend right,line width=0.1cm},
        v6/v2/2/{left},
        v6/t/1/{sloped,bend right,below},
        t/v6/2/{sloped,bend right,below},
        v4/t/1/{left,bend right, pos=.4,line width=0.1cm},
        t/v4/4/{left,bend right, pos=.4},
        t/v5/4/{sloped, above},
        v3/v5/3/{left,pos=.2,bend right},
        v5/v3/1/{right}%
      }{
        \draw[fed] (\u) edge[apply style/.expand once=\style] 
          node{\f} (\v);
      }
    \end{scope}
  \end{tikzpicture}
  \caption{Example flow network.}
  \label{img:flow}
\end{figure}

The code of \cref{img:flow} not only demonstrate how to draw graph, but also how to use for loops and some useful edge style modifiers.
Definitely worth checking out.

\section{Lorem Ipsums}

\subsection {Chinese lorem}

\sout{顏色}香半,褕海知,舉臥之斯文大…\textit{萬里送,Lorem ipsum dol'or} sit amet, consectetuer adipiscing elit. 幾度九華帳山色之難難,楚能余亦之難難夜日行陽為君。萬事仙臥月南風何十怨遙夜⋯風沙,師落葉滿秋,歲閣自照露,征柳啾啾月一曲然不得有殷勤,遠不到酒稀清輝。腸何時還,行路難獨夜南⋯羅微鳳路雪虛征戰。鳴看煙,衣裳多斜馬,相見秋松下孤城君不見西羽歲王孫:見臨烽火桃李但見茫茫獨,問見得城月涕淚長可聖兒夢不成,掩至今歌。雖識月明如此,曲夢在新到天秋一,寂昔不逢闌干,窗東流水腸斷角不見清天與故人,下兒雨千門微遲里劍閣白,石鼓。

\blindtext

\subsection{Ipsum}

\blindtext

沒有好棒,別嗚哈哈哈,都這個實說不然也不,的人的什麼最好。經不院可以去為什麼有,沒該會人嗎亡如果是,再次存跟我試圖,只有直接說這是放到最近的的影片⋯陌生人⋯也知道個月的。其覺得是起來跟不可能也很,畫面待喔好天的簡單很不是聽起來,然已經就聖我的話的聲。再然後道了嗎傍晚時害的,男的話就需要一,或許是的男,去的方法我愛次自行要跟⋯了沒好會的角色這邊,一把的人會出們看記⋯

\end{document}
